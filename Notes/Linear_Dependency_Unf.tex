\documentclass{article}
\usepackage[utf8]{inputenc}
\usepackage{amssymb}
\usepackage{amsmath}

\title{Linear Dependency Relating to Span}
\author{Jaken Herman }
\date{January 2016}

\begin{document}

\maketitle

\section*{Recall}
The \textbf{span} of vectors $\vec{v}_{\,1}, \vec{v}_{\,2}, \dots , \vec{v}_{\,n} \in \mathbb{R}^{k}$ is span(\vec{v}_{\,1}, \vec{v}_{\,2}, \dots , \vec{v}_{\,n}) $ = \{x_1}\vec{v}_{\,1}, x_2\vec{v}_{\,2}, \dots , x_n\vec{v}_{\,n} : x_1, x_2, \dots , x_n \in \mathbb{R} \}$.
\newline

Last time, we said: 

 \begin{itemize}
     \item    \indent span($\vec{v}) = \{x_1\vec{v}_{\, 1} : x_1 \in \mathbb{R} \}$, which is a line (provided $\vec{v} \neq \vec{0}$) containing $\vec{0}$, in the same direction as $\vec{v}$.
     \item \indent If $\vec{v}_{\,1}$ and $\vec{v}_{\,2}$ are nonzero vectors not in the same line then span($\vec{v}_{\,1}, \vec{v}_{\,2}$) is the plane containing $\vec{v}_{\,1}$, $\vec{v}_{\,2}$, and $\vec{0}$.
 \end{itemize}


\textbf{Ex.)}
In $\mathbb{R}^3$, the equation of a plane containing $\vec{0}$ is $$ ax_1 + bx_2 + cx_3 = 0 $$ where at least one of $a, b, c \neq 0$. Assume $a \neq 0$. Then, to describe the solution set (i.e., plane) of $ ax_1 + bx_2 + cx_3 = 0 $, look at 

\[
\begin{bmatrix}
a & b & c & 0
\end{bmatrix} \mathrel{\mathop{\rightarrow}^{\mathrm{R_{1} \rightarrow (\dfrac{1}{a}R_{1})}}
\begin{bmatrix}
1  & $ \dfrac{b}{a} $ & $ \dfrac{c}{a} $ & 0
\end{bmatrix}
\]

which represents $$x_1 + \dfrac{b}{a}x_2 + \dfrac{c}{a}x_3 = 0$$ which is $$x_1 = -\dfrac{b}{a}x_2 -\dfrac{c}{a}x_3$$ so the solution set is:

\[ 
\left \{
 \left \Biggg[
  \begin{tabular}{ccc}
  $-\dfrac{b}{a}x_2 -\dfrac{c}{a}x_3$ \\
  x_2 \\
  x_{3}
  \end{tabular}
  \right \Biggg] : x_{2}, x_{3} \in \mathbb{R}
\right \}.
\]

If we look at the information solely inside the brackets (not the braces), we can view that as



\[ 
\left \{
 x_2  \left \Biggg[
  \begin{tabular}{ccc}
  $-\dfrac{b}{a}$ \\
  1 \\
  0
  \end{tabular}
  \right \Biggg]
  + x_3
   \left \Biggg[
  \begin{tabular}{ccc}
  $-\dfrac{c}{a}$ \\
  0 \\
  1
  \end{tabular}
  \right \Biggg] : x_2, x_3 \in \mathbb{R}
\right \} = span(\left \Biggg[
  \begin{tabular}{ccc}
  $-\dfrac{b}{a}$ \\
  1 \\
  0
  \end{tabular}
  \right \Biggg],
   \left \Biggg[
  \begin{tabular}{ccc}
  $-\dfrac{c}{a}$ \\
  0 \\
  1
  \end{tabular}
  \right \Biggg])
\]

so every plane in $\mathbb{R}^3$ containing the origin is the span of 2 vectors.

\subsection*{Def: }
Let $\vec{v}_{\,1}, \vec{v}_{\,2}, \dots , \vec{v}_{\,n}$ be vectors in $\mathbb{R}^k$. We say these vectors span $\mathbb{R}^k$ if \textbf{every} $\vec{b} \in \mathbb{R}^k$ is in span($\vec{v}_{\,1}, \vec{v}_{\,2}, \dots , \vec{v}_{\,n}$). That is, if every $\vec{b} \in \mathbb{R}^k$ can be writen as a linear combination of $\vec{v}_{\,1}, \vec{v}_{\,2}, \dots , \vec{v}_{\,n}$.

\hfill

\textbf{Ex. 2)}
Do \dononbreakablespace 
\begin{bmatrix}
1 \\ 0 \\ 0
\end{bmatrix},
\begin{bmatrix}
1 \\ 1 \\ 1
\end{bmatrix}, 
\begin{bmatrix}
2 \\ 1 \\ -1
\end{bmatrix}
 \dononbreakablespace span $\mathbb{R}^3$?
 

\textbf{Solution: )}
Restating, if $\vec{b} = \begin{bmatrix} b_1 \\ b_2 \\ b_3 \end{bmatrix}$ is \textbf{any} vector in $\mathbb{R}^3$, do there exist weights $x_1, x_2, x_3$ such that $$x_1 \begin{bmatrix} 1 \\ 0 \\ 0 \end{bmatrix} + x_2 \begin{bmatrix} 1 \\ 1 \\ 1 \end{bmatrix} + x_3 \begin{bmatrix} 2 \\ 1 \\ -1 \end{bmatrix} = \begin{bmatrix} b_1 \\ b_2 \\ b_3 \end{bmatrix}?$$
 
That is, $$ x_1 + x_2 + 2x_3 = b_1$$ $$x_2 + x_3 = b_2$$ $$x_2 - x_3 = b_3$$ 

We will call this system (*) for future reference. So we try to solve this:

\end{document}
